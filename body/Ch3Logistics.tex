\section{逻辑回归分类器实现}

\subsection{逻辑回归实现多分类原理}
通常,主要使用逻辑回归解决二分类的问题,但同样也可以使用逻辑回归解决多分类问题。
\subsubsection{one vs rest}
由于概率函数$h_{\theta}(x)$所表示的是样本标记为某一类型的概率,但可以将一对一(二分类)扩展为一对多(one vs rest):
\begin{enumerate}
  \item 将类型class1看作正样本,其他类型全部看作负样本,然后我们就可以得到样本标记类型为该类型的概率p1;
  \item 然后再将另外类型class2看作正样本,其他类型全部看作负样本,同理得到p2;
  \item 以此循环,我们可以得到该待预测样本的标记类型分别为类型class i时的概率pi,最后我们取pi中最大的那个概率对应
  的样本标记类型作为我们的待预测样本类型。
\end{enumerate}

\subsubsection{softmax函数}
使用softmax函数构造模型解决多分类问题。
softmax回归分类器需要学习的函数为 :
\begin{equation}
  h_\omega (\vec x)=\frac{1}{\sum_{i=1}^k e^{\vec{\omega_i}\cdot\vec{x}+b_i}}\begin{bmatrix}
    \vec{\omega_1}\cdot\vec x+b_1     \\
    \vec{\omega_2}\cdot\vec x+b_2 \\
    \vdots \\
    \vec{\omega_k}\cdot\vec x+b_k
  \end{bmatrix}
\end{equation}
其中k个类别的个数这里写$\vec{\omega_i}$和$b_i$为第$i$个类别对应的权重向量和偏移标量。
其中$\frac{\vec{\omega_j}\cdot\vec{x}+b_j}{\sum_{i=1}^k e^{\vec{\omega_i}\cdot\vec{x}+b_i}}=P(y=j|\vec{x})$可看作样本X的标签为第$j$个类别的概率,且有$\sum_{j=1}^{k}P(y=j|\vec{x})=1$。

与logistic回归不同的是,softmax回归分类模型会有多个的输出,且输出个数与类别个数相等,输出为样本X为各个类别的概率,最后对样本进行预测的类型为概率最高的那个类别。

我们需要通过学习得到$\vec{\omega_i}$和$b_i$,因此建立目标损失函数为:
\begin{equation}
  J(\omega,b)=-\frac{1}{m}\sum_{j=1}^{m}\sum_{l=1}^{k}1\{y^{(j)}=l\}log(\frac{e^{\vec{\omega_l}\vec{x^{(j)}}+b_l}}{\sum_{i=1}^{k}e^{\vec{\omega_i}\vec{x^{(j)}}+b_i}})
\end{equation}
上式的代价函数也称作:对数似然代价函数。在二分类的情况下,对数似然代价函数 可以转化为 交叉熵代价函数。

其中 $m$ 为训练集样本的个数,$k$ 为 类别的个数,$1\{\cdot\}$为示性函数,当$y^{(j)}=l$ 为真时,函数值为 1 ,否则为 0 ,即 样本类别正确时,函数值才为 1 。

通过对数性质以及梯度下降法和链式偏导,化简得:
\begin{equation}
  \frac{\partial{J(\omega,b)}}{\partial\vec{\omega_r}}=-\frac{1}{m}\sum_{j=1}^m\left(\vec{x^{(j)}}-P(y^{(i)}=r|x^{(j)})*\vec{x^{(j)}}\right)
\end{equation}
因此由梯度下降法进行迭代:
\begin{equation}
  \vec{\omega_r}=\vec{\omega_r}-\alpha\frac{\partial{J(\omega,b)}}{\partial\vec{\omega_r}}
\end{equation}
同理 通过梯度下降法最小化损失函数也可以得到$b_i$的最优值。同逻辑回归一样,可以给损失函数加上正则化项。

当标签类别之间是互斥时,适合选择softmax回归分类器 ;当标签类别之间不完全互斥时,适合选择建立多个独立的logistic回归分类器。由于Mnist数据集类别间互斥,故选择softmax回归分类器。

\subsection{实现softmax回归分类器}
由于最近在学习tensorflow,想借用作业机会,运用tensorflow实现softmax回归分类器。
\subsubsection{加载数据}
tensorflow内部集成了Mnist数据集,并且label文件格式是每个样本为一维10元素数组,若为哪一类则对应为1。
\begin{python}
  #Mnist数据集
  from tensorflow.examples.tutorials.mnist import input_data
  mnist=input_data.read_data_sets('data/',one_hot=True)
  trainimg=mnist.train.images #28*28*1
  trainlabel=mnist.train.labels#1*10
  testimg=mnist.test.images#28*28*1
  testlabel=mnist.test.labels#1*10
\end{python}
\subsubsection{softmax分类器实现}
分类器参数设置。
\begin{python}
  #逻辑回归
  #参数设置
  numClasses=10
  inputSize=784
  trainingIterations=50000
  batchSize=64#设置batch大小
  #指定x和y大小
  X=tf.placeholder(tf.float32, shape=[None,inputSize])
  y=tf.placeholder(tf.float32,shape=[None,numClasses])
  #参数初始化
  W1=tf.Variable(tf.random_normal([inputSize,numClasses],stddev=0))
  B1=tf.Variable(tf.constant(0.1),[numClasses])
\end{python}

构建模型。
\begin{python}
  #构造模型
  y_pred=tf.nn.softmax(tf.matmul(X,W1)+B1)

  loss=tf.reduce_mean(tf.square(y-y_pred))
  opt=tf.train.GradientDescentOptimizer(learning_rate=0.05).minimize(loss)

  correct_prediction=tf.equal(tf.argmax(y_pred,1),tf.argmax(y,1))
  accuracy=tf.reduce_mean(tf.cast(correct_prediction,"float"))

  sess=tf.Session()
  init=tf.global_variables_initializer()
  sess.run(init)
\end{python}

迭代计算。
\begin{python}
  #迭代计算
  for i in range(trainingIterations):
      batch=mnist.train.next_batch(batchSize)
      batchInput=batch[0]
      batchLabels=batch[1]
      _,trainingLoss=sess.run([opt,loss],feed_dict={X:batchInput,y:batchLabels})
      if i%1000==0:
          train_accuracy=accuracy.eval(session=sess,feed_dict={X:batchInput,y:batchLabels})
          print("step %d, training accuracy %g"%(i,train_accuracy))
\end{python}

\subsection{测试结果与分析}
\begin{python}
  #测试结果
  batch=mnist.test.next_batch(batchSize)
  testAccuracy=sess.run(accuracy,feed_dict={X:batch[0],y:batch[1]})
  print("test accuracy %g"%(testAccuracy))
\end{python}
最终得到模型准确率为$90.5\%$。


