\section{朴素贝叶斯分类器实现}
\subsection{朴素贝叶斯(Naive Bayesian, NB)分类器原理}
NB分类算法是概率学派的经典算法,也是机器学习中的一个非常经典非常基础的分类算法,NB算法有很强的数学
理论作为支撑。

假设我们有一组训练数据:
$$\{(x_1,c_1),(x_2,c_2),\cdots,(x_n,c_n)\}$$
其中ci是每个数据的标签类别,xi是特征向量,我们需要根据这些训练数据训练一个朴素贝叶斯分类器来对新
数据进行分类。

既然是贝叶斯学派的算法,我们首先不得不引入贝叶斯公式,这也是算法的基础:
\begin{equation}
  \label{eq:bayesBase}
  P(c|x)=\frac{P(c)P(x|c)}{P(x)}
\end{equation}
这里可以进行理解: 
\begin{enumerate}
  \item $P(c|x)$:对于给定特征属性x条件下,数据属于c类别的概率
  \item $P(c)$:样本空间中各个类别所占的比例
  \item $P(x|c)$:对于给定类别的条件下的特征属性组合是x的概率
\end{enumerate}

然而$P(x|c)P(x|c)$是不能直接求得的,因为这里涉及到x的联合概率,即$P(x|c)=P(x_1,x_2,\cdots,x_n|c)P(x|c)$,直接估计每个类别c下的特征属性x的频率是不准确的,
首先$x$的排列组合空间数可能会非常大,会引起组合爆炸问题,其次$x$的所有排列组合中有很多情况是
样本数据中并没有出现的,这就引起了数据稀疏的问题,所以$P(x|c)P(x|c)$是不能直接进行求解
的。

为了解决上述问题,于是,在NB算法中作出一个非常重要的假设,那就是数据的每个属性是条件独立的,
因为属性是相互独立的,那么我们就不需要使用联合概率来表示上面的
$P(x|c)=P(x_1,x_2,\cdots,x_n|c)$了,
我们可以得到新的公式:
\begin{equation}
  \label{eq:bayesBaseNew}
  P(c|x)=\frac{P(c)P(x|c)}{P(x)}=\frac{P(c)}{P(x)}\prod_{i=1}^d P(x_i|c)
\end{equation}

我们可以看到,因为有了属性条件独立的假设,$P(x|c)=\prod_{i=1}^d P(x_i|c)$,
$d$为属性个数。这样我们 就很好进行计算了,只需要分别统计每个类别下的属性取值所占比例即可。
这也就是我们的似然函数,我们可以将$\prod_{i=1}^d P(x_i|c)$取对将连乘转换为累加,避免数据下溢,
来进行计算\upcite{西瓜书}。

\subsection{朴素贝叶斯分类器实现}
\subsubsection{数据预处理}
从\ref{Ch:Mnist}我们已经从Mnist数据集读取了我们训练分类器所需要的图片和标签数据,
然而得到的图片是28*28的数组,为了便于后续计算,我们需要将二维数组转为一维数组。
\begin{python}
  # 训练集图片
  train_input=train_images.reshape((train_images.shape[0],train_images.shape[1]*train_images.shape[2]),order='F')
  #测试集图片
  test_input=test_images.reshape((test_images.shape[0],test_images.shape[1]*test_images.shape[2]),order='F')
\end{python}

将图像进行二值化处理,以减少了属性量,从而减少了运算量。
\begin{python}
  def normalize(data):
    m=data.shape[0]
    n=np.array(data).shape[1]
    for i in range(m):
        for j in range(n):
            if data[i,j]!=0:
                data[i,j]=1
    return data
\end{python}

\subsubsection{NB分类器实现}
首先计算先验概率和条件概率。
\begin{python}
  def BayesModel(traindata,trainlabel,classnum):
    traindatasize=traindata.shape[0]
    samplesize=traindata.shape[1]
    
    #先验概率
    PriorProbability=np.zeros(classnum)
    #条件概率
    ConditionalProbability=np.zeros((classnum,samplesize,2))
    #计算先验概率和条件概率
    for i in range(traindatasize):
        sample=traindata[i]
        label=int(trainlabel[i])
        #统计label类的label数量
        PriorProbability[label]+=1
        for j in range(samplesize):
            temp=int(sample[j])
            ConditionalProbability[label][j][temp]+=1
    
    #将概率归到[1.10001]
    for i in range(classnum):
        for j in range(samplesize):
            #经过二值化的图像只有0,1两种取值
            pix0=ConditionalProbability[i][j][0]
            pix1=ConditionalProbability[i][j][1]

            #计算0,1像素点对应的条件概率
            probability0=(float(pix0)/float(pix0+pix1))*10000+1
            probability1 = (float(pix1)/float(pix0 + pix1)) * 10000 + 1

            ConditionalProbability[i][j][0]=probability0
            ConditionalProbability[i][j][1]=probability1
    
    return PriorProbability,ConditionalProbability
\end{python}

对于给定的测试样本,可以计算先验概率与条件概率的乘积。
\begin{python}
  def calProbability(sample,label,PriorProbability,ConditionalProbability):
    probability=int(PriorProbability[label])#先验概率
    samplesize=sample.shape[0]
    for i in range(samplesize):#应该是特征数
        probability*=int(ConditionalProbability[label][i][sample[i].astype(int)])
    return probability
\end{python}

这样就可以确定该样本所属的分类,相当于做了argmax运算。
\begin{python}
  def predict(testdata,testlabel,PriorProbability,ConditionalProbability,classnum):
    predictLabel=[]
    testdatasize=testdata.shape[0]
    testsamplesize=testdata.shape[1]
    for i in range(testdatasize):
        sample=np.array(testdata[i])
        label=testlabel[i]
        maxlabel=0
        maxprobability=calProbability(sample,0,PriorProbability,ConditionalProbability)
        for j in range(1,classnum):
            probability=calProbability(sample,j,PriorProbability,ConditionalProbability)
            if(maxprobability<probability):
                maxprobability=probability
                maxlabel=j
        predictLabel.append(maxlabel)
    return np.array(predictLabel)
\end{python}

\subsection{分类器准确性分析}
\subsubsection{计算准确率}
\begin{python}
  def calAccuracy(testlabel,predictlabel):
    testdatasize=testlabel.shape[0]
    errorcount=0
    for i in range(testdatasize):
        if(testlabel[i]!=predictlabel[i]):
            errorcount+=1
    accuracy=1.0-float(errorcount)/testdatasize
    return accuracy
\end{python}
\subsubsection{分析}
最终训练出的分类器的准确率约为$84.5\%$,可以发现远低于kNN分类器的$96.5\%$的分类准确性,这可能由于
是对数据进行二值化处理导致的,然而如果不进行二值化处理,计算量过大,本人计算机跑不出来。

